\documentclass[a4paper, 12pt]{bxjsarticle}
\usepackage{graphicx}
\usepackage{amsmath}
\usepackage{txfonts}
\usepackage{siunitx}
\usepackage{mathspec}
\setmainfont{IPAexMincho}
\setsansfont{IPAexGothic}
\XeTeXlinebreaklocale "ja"
\topmargin = 0mm
\pagestyle{empty}

\renewcommand{\thesubsection}{\arabic{subsection}.}
\renewcommand{\thesubsubsection}{(\alph{subsubsection})}

\begin {document}

\begin{center}
    \begin{huge}
        電磁気学 11/13 宿題
    \end{huge}
\end{center}

% p.61 1.13
\subsection{電気容量が \(C_1\) と \(C_2\) の2つのコンデンサーがある.初めに容量 \(C_1\) のコンデンサーを電位差 \(V\) で充電する.
このとき容量 \(C_1\) のコンデンサーに蓄えられる静電エネルギーを求めよ.
次に,これを電源から切り放し,充電されていないもう一方のコンデンサーに並列につなぐと全体の静電エネルギーはどうなるか(p.61 図1.45(b) 参照).
特に,\(C_1 = 0.01\si{[\mu F]},\;C_2 = 2.5\si{[\pico F]},\;V=100\si{[V]}\) のときに,最終状態の静電エネルギー値はどうなるか.}

\newpage
% p.61 1.14
\begin{samepage}
\subsection{p.62 図 1.46(a)のように,面積 \(S\) の\(2\)枚の薄い導体極板を間隔が \(d_1\) になるように平行に置き,一方に電荷 \(Q\) を他方に \(-Q\) の電荷を与える.以下の問に答えよ.ただし,導体上の電荷密度は場所によらず一定であり,電界は両導体間にのみ存在するものとする.}
\subsubsection{両極板の引き合う力を求めよ.}
\vspace{13em}
\subsubsection{極板に力を加えて間隔を \(d_2\) まで拡げるとき,どれだけの仕事が必要か.}
\vspace{13em}
\subsubsection{p.62 図 1.46(b)のように,両極板間の電位差を常に \(V\) に保つようにして,極板間の間隔を \(d_1\) から \(d_1\) まで変化させるとき,どれだけの仕事が必要か.}
\end{samepage}
\newpage

% テキスト p.16 問2
\subsection{次の電界のエネルギー密度に関する式を示せ.\[u=\frac{1}{2}\varepsilon E^2 = \frac{1}{2} D E\;\;\;\;\si{(J/m^3)}\]}

\end{document}

