\documentclass[a4paper, 12pt]{bxjsarticle}
\usepackage{graphicx}
\usepackage{amsmath}
\usepackage{txfonts}
\usepackage{siunitx}
\usepackage{mathspec}
\setmainfont{IPAexMincho}
\setsansfont{IPAexGothic}
\XeTeXlinebreaklocale "ja"
\topmargin = 0mm
\pagestyle{empty}

\renewcommand{\thesubsection}{\arabic{subsection}.}
\renewcommand{\thesubsubsection}{(\alph{subsubsection})}

\begin {document}

\begin{center}
    \begin{huge}
        電磁気学 12/11 宿題
    \end{huge}
\end{center}

% p.98 問2.13
\subsection{半径 \(a\) の無限に長い円柱の中を中心軸に沿って電流 \(I\) が一様な密度で流れている.中心軸から距離 \(r\) の点での磁束密度を求めよ.}
% p.102 2.10
\subsection{p.102 図2.35(b)のように,半径 \(a\) の半円と \(2\) 本の無限に長い直線とからなる導線に,電流 \(I\) を流す.円弧の中心 \(\mathrm{O}\) における磁束密度を求めよ.}
% p.103 2.16
\subsection{p.103 図2.36(b)のように,導線をドーナツ状に巻いたコイルをトロイダルコイル(troidal coil)という.半径 \(R\),太さ \(2a\),総巻数 \(N\) のトロイダルコイルに電流 \(I\) を流す.%
トロイダルコイル内で中心 \(\mathrm{O}\) から距離 \(r\) の点での磁束密度を求めよ.また,\(R \gg a\) のときに磁束密度はどうなるか.}
% p.104 2.18
\subsection{p.104 図2.37(b)のように,\(yz\) 平面に平行な \(2\) 枚の無限に広い平面状導体の上を,電流が単位長さ当たり \(J\) の大きさで,\(z\) 軸に平行で互いに逆向きに流れている.まわりの空間での磁束密度を求めよ.}

\end{document}

