\documentclass[a4j,12pt]{jsarticle}
\usepackage{amsmath}
\usepackage{txfonts}
\topmargin = 0mm
\pagestyle{empty}

\renewcommand{\thesubsection}{\arabic{subsection}.}
\renewcommand{\thesubsubsection}{(\roman{subsubsection})}

\makeatletter
    \renewcommand{\theequation}{\arabic{subsection}.\arabic{equation}}
    \@addtoreset{equation}{subsection}
\makeatother

\begin {document}

\begin{center}
    \begin{LARGE}
        {\huge 電磁気学 10/2 宿題}\vspace{0.2em}\\ 416841 長谷川央
    \end{LARGE}
\end{center}

\subsection{常温の銅では銅原子の最外殻電子のうちの\(1個\)が原子から離脱して自由電子になる.\(1g\)の銅内の自由電子数はいくらか.}

\vspace{20em}
\subsection{距離\(1[\AA]\)を隔てた2個の電子の間に働くクーロン力と万有引力の大きさを求めよ.ただし,電子の質量と電荷はそれぞれ \(m_e=9.11\times10^{-31}[kg], -e=-1.60\times10^{-19}[C]\)である.また万有引力定数は\(G=6.67\times10^{-11}[Nm^2/kg^2]\)である.}
\end{document}
