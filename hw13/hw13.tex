\documentclass[a4paper, 12pt]{bxjsarticle}
\usepackage{graphicx}
\usepackage{amsmath}
\usepackage{txfonts}
\usepackage{siunitx}
\usepackage{mathspec}
\setmainfont{IPAexMincho}
\setsansfont{IPAexGothic}
\XeTeXlinebreaklocale "ja"
\topmargin = 0mm
\pagestyle{empty}

\renewcommand{\thesubsection}{\arabic{subsection}.}
\renewcommand{\thesubsubsection}{(\alph{subsubsection})}

\begin {document}

\begin{center}
    \begin{huge}
        電磁気学 1/15 宿題
    \end{huge}
\end{center}

\begin{samepage}
    \subsection{直線上の導線が地球赤道上で東西方向に地面と平行に置かれている.この点での地磁気は水平であって,磁束密度 \(B = 3.3\times10^{-5}\;(\si{T})\) である.%
導線の単位長さ当たりの質量が \(2.0\times10^{-3}\;(\si{kg/m})\) であるとき,導線の重量を磁気力で支持するためには,導線にどれほどの電流を流さなければならないか?%
(答:東向きに \(594\si{A}\))}
\vspace*{10em}

    \subsection{p.103 演習問題 2.16でトロイダルコイルの外半径 \(R =1.3\;\si{(m)}\),内半径 \(a = 0.7\;\si{(m)}\),総巻数 \(N=900\) 回である.このコイルに電流 \(I=14000\si{A}\) を%
流すとき,次の位置での磁束密度の大きさを計算せよ.}

\subsubsection{トロイドの内径部}

    \vspace*{10em}

\subsubsection{トロイドの外径部}

\end{samepage}
\newpage
\begin{samepage}
\subsection{p.108 図3.3の装置で,\(\mathrm{A}\) 点間,\(\mathrm{B}\) 点間を \(R=6\si{\Omega}\) の抵抗で結ぶ.\(l=1.2\si{m}\),磁束密度 \(B=2.5\si{T}\) である.}
\subsubsection{棒 \(\mathrm{CD}\) を定速 \(v=2\si{m/s}\) で右に運動させるのに必要な外力を求めよ.(答:右向きに \(3.00\si{N}\))}
\vspace*{7em}
\subsubsection{抵抗内での消費電力はいくらか?(答:\(6.00\si{W}\))}
\vspace*{7em}
\subsection{次の交流発電機では,巻線数 \(N=8\),各巻線の面積 \(S=0.09\si{m^2}\),巻線の全抵抗 \(r=12\si{\Omega}\) である.ループは磁束密度 \(B=0.5\si{T}\)の中で\(60\si{Hz}\)で回転する.%
発電機の出力端子に外部抵抗 \(R=20\si{\Omega}\) を接続する.次の量を求めよ.}
\subsubsection{誘導起電力の時間変化の式}
    \vspace*{7em}
\subsubsection{回路電流の時間変化の式}
    \vspace*{7em}
\subsubsection{発電機の消費電力と外部抵抗の消費電力}
    \vspace*{7em}
\end{samepage}
\newpage
\begin{samepage}
%p.135 3.4
\subsection{p.135 図3.15のように,\(x=0\) と \(x=l\) の境界面に挟まれた領域にだけ磁束密度 \(B\) の一様な磁界が \(z\) 方向正の向きに加えられている.\(xy\) 面内に,一辺の長さが \(a\) と \(b\) の長方形コイル%
 \(\mathrm{ABCD}\) が辺 \(\mathrm{AD}\) と境界面が平行になるように置いてある.このコイルを一定の速さ \(v\) で \(x\) 軸方向に動かす.辺 \(\mathrm{AD}\) が \(x=0\) に達したときを時刻 \(t=0\) として,コイルに発生する%
 誘導起電力を \(t\) の関数として求めよ.ただし \(a<l\) とする.}
\vspace*{12em}
%p.136 3.6
\subsection{導線をドーナツ状に巻いた半径 \(R\),太さ \(2a\),総巻数 \(N\) のトロイダルコイルがある(p.103 図2.36(b) 参照).以下の問に答えよ.ただし,\(R\gg a\) とする.}
\subsubsection{このトロイダルコイルの自己インダクタンスを求めよ.}
\vspace*{6em}
\subsubsection{このトロイダルコイルに電流 \(I\) を流すとき,コイルに蓄えられる磁気エネルギーを求めよ.}
\vspace*{6em}
\subsubsection{磁気エネルギーが磁界中に蓄えられるとして,磁気エネルギーを求め,(2)で求めた値と等しくなることを確かめよ.}
\vspace*{6em}
\end{samepage}
\newpage
\subsection{コイル \(\#1\) とコイル \(\#2\) があり,それぞれに電流 \(I_1\) と \(I_2\) が流れ,それぞれに鎖交する磁束が \(\Phi_1, \Phi_2\) である.磁界のエネルギーが, \(\frac{1}{2} I_1 \Phi_1 + \frac{1}{2} I_2 \Phi_2\; \si(J)\) であることを示せ.(ヒント:テキストp.29)}
\end{document}

