\documentclass[a4paper, 11pt]{bxjsarticle}
\usepackage{graphicx}
\usepackage{amsmath}
\usepackage{txfonts}
\usepackage{siunitx}
\usepackage{mathspec}
\setmainfont{IPAexMincho}
\setsansfont{IPAexGothic}
\XeTeXlinebreaklocale "ja"
\addtolength{\textheight}{1cm}
\addtolength{\topmargin}{-0.5cm}
\addtolength{\textwidth}{2cm}
\addtolength{\oddsidemargin}{-1cm}
\addtolength{\evensidemargin}{-1cm}
\topmargin = 0mm
\pagestyle{empty}

\renewcommand{\thesubsection}{(\arabic{subsection})}
\renewcommand{\thesubsubsection}{(\roman{subsubsection})}

\begin {document}
\begin{samepage}

\begin{center}
    \begin{huge}
        電磁気学 1/22 宿題
    \end{huge}
\end{center}

\section{}
\subsection{教科書 p.127の電磁波に関する記述で,水のような誘電体では,%
\(\rho = 0,\;\boldsymbol{J} = 0\) 及び  \(\boldsymbol{D}=\varepsilon  \boldsymbol{E} =\varepsilon _r \varepsilon _0 \boldsymbol{E}\)%
 (但し,\(\varepsilon _r\) は比誘電率)により,アンペール・マクスウェルの方程式(3.45)は,
\[\mathrm{rot} \boldsymbol{B} = \varepsilon _r \varepsilon _0 \mu_o \frac{\partial \boldsymbol{E}}{\partial t}\]に変わる.このことを示せ.}
\vspace*{20em}
\subsection{上の結果によれば,教科書p.127〜p.130の記述は誘電体に対して,\(\varepsilon _0 \rightarrow \varepsilon_r \varepsilon_0\) と置き換えれば良いことが分かる.}
\end{samepage}
\newpage


\begin{samepage}
\section{}
\subsection{ある平面電磁波の電界が \(E_x=100sin(10^7 t - \omega t)\;\si{(V/m)}\) である.以下の値を求めよ.}
\subsubsection{磁束密度 \(B_y\)}
\vspace*{7.5em}
\subsubsection{波長 \(\lambda\)}
\vspace*{7.5em}
\subsubsection{周波数 \(f\)}
\vspace*{7.5em}

\subsection{このことから,誘電体では電磁波の伝搬速度 \(v\) が真空中の光速 \(c\) に対し,\[v=\frac{1}{\sqrt \varepsilon_r} c\]であることを示せ.}
\vspace*{10em}
\subsection{比誘電率 \(\varepsilon_r = 1.78\) の水中での光速はいくらか.}
\end{samepage}
\newpage

\begin{samepage}
\section{晴れた空の下での日光の強さが \(1000\si{W/m^2}\) であるとき,地表付近の \(1\si{m^3}\) にはどれだけ電磁エネルギーが含まれているか?}
\vspace*{25em}
\section{ある電磁波の磁界の振幅が\[B_m = 4.1\times 10^{-8}\;\si{(T)}\]である.電磁波の強度 \(<S>\) を求めよ.(Hint. [教] p.134 (3.78))}
\end{samepage}

\newpage

\begin{samepage}
\section{ある白熱灯のフィラメントは \(150\si{\Omega}\) の抵抗を持ち,\(1\si{A}\) の直流電流が流れている.フィラメントは長さ \(8\si{cm}\),半径 \(0.9\si{mm}\) である.}
\subsection{フィラメント表面でのポインティングベクトルを求めよ.}
\vspace*{25em}
\subsection{フィラメント表面での電界 \(E\) と磁束密度 \(B\) の強さを求めよ.}
\end{samepage}

\newpage

\begin{samepage}
\section{テキストp40の図17.2は同一平面内にある細い \(2\) 本の平行導線である.導線の抵抗は無視できるとする.これは%
簡単な電話線と考えてよい.線間の電圧を \(V=V_0 \cos \omega t\),導線の電流を \(I = I_0 \cos \omega t\) とする.}
\subsection{送信端(電源)から受信端(抵抗・電話器)に送られる電力の時間平均値(実効値)を示せ.}
\vspace*{13em}
\subsection{\(2\) 本の導線を含む平面内に,\(2\) 導線に垂直な直線を引く.この直線上で,点 \(\mathrm{A}\),点 \(\mathrm{B}\),点 \(\mathrm{C}\) はそれぞれ,テキストp40の図17.2に示す位置にある.}
\subsubsection{\(\omega t = 0\) に於ける電界 \(\boldsymbol{E}\),磁束密度 \(\boldsymbol{B}\),情報の流れる向き(ポインティングベクトル \(\boldsymbol S\) の向き) を \(\mathrm{A}\),\(\mathrm{B}\),\(\mathrm{C}\) の各点について,テキストp40 (2.1)に指定された \(3\) つの記号などを使って,テキストp40 図17.2の上に示せ.\label{ue}}
\vspace*{13em}
\subsubsection{\(\omega t = \pi\) の時刻について,\ref{ue}と同じ問に答えよ.}
\end{samepage}

\end{document}

