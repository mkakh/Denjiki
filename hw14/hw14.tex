\documentclass[a4paper, 12pt]{bxjsarticle}
\usepackage{graphicx}
\usepackage{amsmath}
\usepackage{txfonts}
\usepackage{siunitx}
\usepackage{mathspec}
\setmainfont{IPAexMincho}
\setsansfont{IPAexGothic}
\XeTeXlinebreaklocale "ja"
\topmargin = 0mm
\pagestyle{empty}

\renewcommand{\thesubsection}{\arabic{subsection}.}
\renewcommand{\thesubsubsection}{(\alph{subsubsection})}

\begin {document}

\begin{center}
    \begin{huge}
        電磁気学 1/22 宿題
    \end{huge}
\end{center}

\subsection{}
\subsubsection{教科書 p.127の電磁波に関する記述で,水のような誘電体では,%
\(\rho = 0,\;\boldsymbol{J} = 0,\;\boldsymbol{D}=\varepsilon  \boldsymbol{E} =\varepsilon _r \varepsilon _0 \boldsymbol{E}\)%
 (但し,\(\varepsilon _r\) は比誘電率)により,アンペール・マクスウェルの方程式(3.45)は,
\[\mathrm{rot} \boldsymbol{B} = \varepsilon _r \varepsilon _0 \mu_o \frac{\partial \boldsymbol{E}}{\partial t}\]に変わる.このことを示せ.}
\subsubsection{上の結果によれば,教科書p.127~p.130の記述は誘電体に対して,\(\epsilon _0 \rightarrow \epsilon_r \epsilon_0\) と置き換えれば良いことが分かる.}

\subsection{}
\subsubsection{ある平面電磁波の電界が \(E_x=100sin(10^7 t - \omega t)\;\si{(V/m)}\) である.}

\end{document}

