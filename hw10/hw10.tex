\documentclass[a4paper, 12pt]{bxjsarticle}
\usepackage{graphicx}
\usepackage{amsmath}
\usepackage{txfonts}
\usepackage{siunitx}
\usepackage{mathspec}
\setmainfont{IPAexMincho}
\setsansfont{IPAexGothic}
\XeTeXlinebreaklocale "ja"
\topmargin = 0mm
\pagestyle{empty}

\renewcommand{\thesubsection}{\arabic{subsection}.}
\renewcommand{\thesubsubsection}{(\alph{subsubsection})}

\begin {document}

\begin{center}
    \begin{huge}
        電磁気学 12/4 宿題
    \end{huge}
\end{center}

% p.87 問2.9
\subsection{p.87 図2.19のような,単位長さ当たりの巻数が \(n\) の無限に長いソレノイドに電流 \(I\) が流れている.%
p.85 例題2.4.3の結果を用いてコイルの中心軸上の磁束密度を求めよ.}
\vspace{20em}

% p.99 演習問題2.1
\subsection{p.99 図2.31(a)のように,\(y\) 軸上に無限に長い直線導体が置いてあり,\(x\) 軸上の \(x=d\) と%
 \(x=d+l\) の間に長さ \(l\) の直線導体 \(AB\) が置いてある.これらの直線導体にそれぞれ電流 \(I_1\) と \(I_2\) を図のような向きに流すとき,%
直線導体 \(AB\) に作用する力を求めよ.}

\newpage
% テキスト p.4 m26
\subsection{\(\nabla\cdot\nabla\times \boldsymbol{A}=0\)を導け.}
\vspace{10em}
% テキスト p.4 from m19 to m21
\subsection{次の式を導け.}
\vspace*{-2em}
\begin{align}
    \mathrm{grad}\left(\frac{1}{r}\right)=\nabla\left(\frac{1}{r}\right)=-\frac{\hat r}{r^2}
\end{align}
\(r \ne 0\)のとき,
\begin{align}
    \nabla^2\left(\frac{1}{r}\right) = \mathrm{div\;grad}\left(\frac{1}{r}\right) = \nabla\cdot\nabla\left(\frac{1}{r}\right) = 0
\end{align}
\(V\)を点\(\mathrm{P}\)を含む体積としたとき,
\begin{align}
    \int_V \nabla^2 \left(\frac{1}{r}\right) dv'= -4\pi
\end{align}
\newpage
% テキスト p.4 m23
\subsection{\(\nabla\times\nabla\times\boldsymbol{A}=\nabla\left(\nabla\cdot\boldsymbol{A}\right)-\nabla^2\boldsymbol{A}\)を導け.}

\end{document}

