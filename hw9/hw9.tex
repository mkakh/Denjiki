\documentclass[a4paper, 12pt]{bxjsarticle}
\usepackage{graphicx}
\usepackage{amsmath}
\usepackage{txfonts}
\usepackage{siunitx}
\usepackage{mathspec}
\setmainfont{IPAexMincho}
\setsansfont{IPAexGothic}
\XeTeXlinebreaklocale "ja"
\topmargin = 0mm
\pagestyle{empty}

\renewcommand{\thesubsection}{\arabic{subsection}.}
\renewcommand{\thesubsubsection}{(\alph{subsubsection})}

\begin {document}

\begin{center}
    \begin{huge}
        電磁気学 11/29 宿題
    \end{huge}
\end{center}

\subsection{電気抵抗率 \(\rho = 1.0\times10^{-6}\si(\Omega \cdot m)\) のニクロム線がある.長さ \(50\si{m}\),直径 \(1\si{mm}\) の%
線と長さ \(25\si{mm}\),直径 \(2\si{mm}\) の線を直列に接続した時の全体の抵抗値はいくらか.}
\vspace{20em}

% p.75 問2.4
\subsection{\(1\) 辺の長さ \(a\) の三角形の各頂点に \(3\) 本の無限に長い直線導線を三角形の面に垂直になるように配置する.各導線に電流 \(I\) をすべて同じ向きに流す場合と,%
\(1\) つおきに逆向きに流す場合とで,導線の単位長さ当たりに働く力を求めよ.}

\newpage
% p.99 2.3
\subsection{磁束密度 \(\boldsymbol{B}\) の一様な磁界の中に電荷 \(q\),質量 \(m\) の荷電粒子が速さ \(v_0\) で \(\boldsymbol{B}\) と \(\theta\) をなす角度で入射したとき,%
その荷電粒子はその後どのような運動をするか.}
\vspace{20em}

% p.77 問2.5
\subsection{電荷 \(1.6\times10^{-19}\si{[C]}\),質量 \(1.7\times10^{-27}\si{[kg]}\) をもつ陽子を磁束密度 \(1.0\si{[T]}\)の一様な磁界中で%
サイクロトロン運動をさせたとき,その角振動数はいくらになるか.また,初速が \(2.0\times10^{5}\si{[m/s]}\) のとき,%
\(r\) (ラーマー半径)はいくらか.}
\newpage

% 掲示板の図より
\subsection{導体の直方体に,直行する磁界・電圧を加える.電圧により駆動される自由電子は \(\boldsymbol{B}\) により曲げられた軌道を取るので,%
電子の右向きの流れが生ずる.その結果,右側の壁に電子が帯電する.しかし,実際にはこの電子の右向きの流れは直ぐに停止してしまい,%
定常状態に達する.なぜ,右向き電子流は停止するのだろうか.}
\end{document}

