\documentclass[a4paper, 12pt]{bxjsarticle}
\usepackage{graphicx}
\usepackage{amsmath}
\usepackage{txfonts}
\usepackage{siunitx}
\usepackage{mathspec}
\setmainfont{IPAexMincho}
\setsansfont{IPAexGothic}
\XeTeXlinebreaklocale "ja"
\topmargin = 0mm
\pagestyle{empty}

\renewcommand{\thesubsection}{\arabic{subsection}.}
\renewcommand{\thesubsubsection}{(\alph{subsubsection})}

\begin {document}

\begin{center}
    \begin{huge}
        電磁気学 12/18 宿題
    \end{huge}
\end{center}

% p.100 2.5
\subsection{p.100 図2.32(a)のように,磁極の内側を円筒の側面のように加工して,磁極近くの磁束密度が常に磁極に垂直で,大きさが \(B\) になるようにした永久磁石がある.%
p.100 図2.32(b)は,磁束線の様子を上から見たものである.この中に,巻数が \(n\) で面積が \(S\) の長方形コイル \(\mathrm{ABCD}\) を置き,コイルの面がはじめ \(\mathrm{N}\) 極と%
\(\mathrm{S}\) 極とを結ぶ線に平行になるように剛性率 \(G\) の細い線で支える.コイルに電流 \(I\) を \(\mathrm{ABCD}\) の順に流すと,コイルに働く力のモーメントとこれを支える細い線内の%
応力による力のモーメントとがつり合って,コイルの元の位置からある角度だけ回転する.このときの回転角 \(\theta\) を求めよ.ただし,コイルの辺 \(\mathrm{AB}\) と \(\mathrm{CD}\) は磁極の近くに%
あるものとする.}
\newpage
\begin{samepage}
% p.100 2.6 (1) (2)
\subsection{p.101 図2.33(a)のような半径 \(a\),質量 \(m\) の円輪の上に,総量 \(q\) の電荷が一様に分布している.この円輪を中心 \(\mathrm{O}\) を通り円輪の面に垂直な軸の周りに角速度 \(\omega\) で%
回転させる.}
\subsubsection{円輪の中心 \(\mathrm{O}\) における磁束密度を求めよ.}
\vspace{10em}
\subsubsection{円輪のもつ磁気モーメントを求めよ.}
\vspace{10em}
% p.102 2.11
\subsection{総量 \(Q\) の電荷が一様に分布する半径 \(a\) の円板を,中心を通り円板面に垂直な軸の周りに角速度 \(\omega\) で回転する.円板の中心での磁束密度を求めよ.}
\vspace{20em}
\end{samepage}
\newpage
\begin{samepage}
% p.103 2.14
\subsection{半径 \(a\) の無限に長い円柱状導体の中を中心軸方向に電流が流れていて,その電流密度は,中心軸からの距離を \(r\) とすると,\[
    J(r) = J\;\frac{a^2-r^2}{a^2}\;\;\;\;\;(r < a)\]で与えられる.}
\subsubsection{導体を流れる電流の総量を求めよ.}
\vspace{10em}
\subsubsection{中心軸から距離 \(r\) の点での磁束密度を求めよ.}
\vspace{10em}
% p.135 3.1
\subsection{地磁気の鉛直成分が \(3.6\times10^{-5}\si{[T]}\) である空中を,両翼の長さが \(50\si{[m]}\) のジェット機が \(1000\si{[km/h]}\) の速さで水平に飛行している.%
翼の両翼間で現れる電位差を求めよ.}
\end{samepage}
\newpage
% p.135 3.2
\subsection{磁束密度 \(B\) の一様な磁界がある.この磁界に垂直な面内で一端 \(\mathrm{O}\) を固定した長さ \(l\) の導体棒 \(\mathrm{OP}\) が置いてあり,\(\mathrm{O}\) を中心に一定の角速度 \(\omega\)で回転している.%
以下の問に答えよ.}
\subsubsection{\label{r1}導体棒 \(\mathrm{OP}\) の中で \(\mathrm{O}\) から距離 \(x\) の点での誘導電界を求め,これを \(x\) について積分することによって,\(\mathrm{OP}\) 間に現れる誘導起電力を求めよ.}
\vspace{15em}
\subsubsection{導体棒 \(\mathrm{OP}\) が時間 \(t\) の間に掃く磁束を求めて,これを時間で微分すると,その大きさが\ref{r1}で求めた誘導起電力に一致することを示せ.}
\vspace{15em}
\subsubsection{\(B=1\si{[T]},\;\omega=100\pi\si{[rad/s]},\;l=1\si{[m]}\)のとき,\(\mathrm{OP}\)間に現れる誘導起電力を計算せよ.}
\vspace{15em}
\newpage
\begin{samepage}
% p.135 3.3
\subsection{\(z\) 方向に振動数 \(v\) で振動する磁界がある.半径 \(a\),巻数 \(n\) の円形コイルをコイルの面が \(xy\) 平面内にあるように置いて,コイルに生ずる誘導起電力を測定したら,その振幅が \(V_0\) であった.%
振動する磁界の磁束密度の振幅 \(B_0\) を求めよ.}
    \vspace{20em}
% オリジナル
\subsection{導線に密に\(100\)回巻いた半径 \(5.0\;\si{cm}\) の円形コイルがある.%
時刻 \(t = 0\) にコイルに垂直に磁界を加える.\(0.1\;\si{s}\) の間に磁束密度を \(0\;\si{T}\) から%
 \(0.5\;\si{T}\) まで直線的に増加する.その間にコイルに生ずる誘導起電力は,何 \(\si{V}\) か.}
\end{samepage}


\end{document}

