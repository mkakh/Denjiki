\documentclass[a4j,12pt]{jsarticle}
\usepackage{amsmath}
\usepackage{txfonts}
\usepackage{siunitx}
\usepackage{bm}
\usepackage{upgreek}
\topmargin = 0mm
\pagestyle{empty}

\renewcommand{\thesubsection}{\arabic{subsection}.}
\renewcommand{\thesubsubsection}{(\alph{subsubsection})}

\begin {document}

\begin{center}
    \begin{LARGE}
        {\huge 電磁気学 10/23 宿題} 
    \end{LARGE}
\end{center}

% p33, 問1.11
\subsection{教科書p.13 図1.8(b)のように,\(x\)軸上に長さ\(2a\)にわたって電荷が一様に線密度\(\uplambda\)で分布している.\(x\)軸上で,原点\(\mathrm{O}\)から\(r\;(r>a)\)だけ離れた点\(\mathrm{P}\)の電位を求めよ.ただし,電位の基準点は無限遠とする.更に,\(\uplambda=1.5\times10^{-9}\si{[C/m]},\;a=5\si{[cm]},\;r=10\si{[cm]}\)の時の\(\phi\)を求めよ.}

\vspace{20em}

% p33, 問1.12
\subsection{半径\(a\)の球全体に総量\(Q\)の電荷が一様に分布している.球の中心\(\mathrm{O}\)から距離\(r\)の点\(\mathrm{P}\)での電位を求めよ.ただし,電位の基準点は無限遠とする.更に,\(Q=2.5\times10^{-10}\si{[C]},\;a=2\si{[cm]},\;r=3\si{[cm]}\)の時の電位を求めよ.}

\end{document}
