\documentclass[a4paper, 12pt]{bxjsarticle}
\usepackage{graphicx}
\usepackage{amsmath}
\usepackage{txfonts}
\usepackage{siunitx}
\usepackage{mathspec}
\setmainfont{IPAexMincho}
\setsansfont{IPAexGothic}
\XeTeXlinebreaklocale "ja"
\topmargin = 0mm
\pagestyle{empty}

\renewcommand{\thesubsection}{\arabic{subsection}.}
\renewcommand{\thesubsubsection}{(\alph{subsubsection})}

\begin {document}

\begin{center}
    \begin{huge}
        電磁気学 11/27 宿題
    \end{huge}
\end{center}

% p.61 1.14
\begin{samepage}
\subsection{p.61 図1.46(a)のように,面積 \(S\) の \(2\) 枚の薄い導体極板を感覚が \(d_1\) になるように平行に置き,一方に電荷 \(Q\) を他方に \(-Q\) の電荷を与える.%
加えて,電極間には誘電率 \(\varepsilon\) の誘電体を挿入してある.%
以下の問に答えよ.ただし,導体上の電荷密度は場所によらず一定であり,電界は両導体間にのみ存在するものとする.}
\subsubsection{両極板の引き合う力を求めよ.}
\vspace{13em}
\subsubsection{極板に力を加えて間隔を \(d_2\) まで拡げるとき,どれだけの仕事が必要か.}
\vspace{13em}
\subsubsection{p.62 図 1.46(b)のように,両極板間の電位差を常に \(V\) に保つようにして,極板間の間隔を \(d_1\) から \(d_1\) まで変化させるとき,どれだけの仕事が必要か.}
\end{samepage}

\newpage
% p.74 式2.26 
\subsection{ローレンツ力は,\(\boldsymbol{F}=q\boldsymbol{v}\times\boldsymbol{B}\) と表せる.このとき,\(x,\;y,\;z\) 成分は\[%
    F_x=q(v_yB_z-v_zB_y),\;F_y=q(v_zB_x-v_xB_z),\;F_z=q(v_xB_y-v_yB_x)\]のように書き表されることを示せ.}
\newpage

% ジュールの法則
\subsection{回路素子に電圧 \(V\) を与え,電流 \(I\) が流れている.\(2\) 点間の電位差 \(\Delta \phi\) は,\(+1\;\si{(C)}\) の電荷を点から点に運ぶのに必要な仕事である.いま,%
時間 \(dt\) 間に \(dq\) の電荷が素子を通過した.その際,電源の供給するエネルギーは,\(dU = Vdq\;\si{(J)}\) である.この電気エネルギーは,素子の中で熱エネルギー,磁気エネルギー%
等に変換される.}
\subsubsection{\(dU=Vdq\) を使って,電源から素子に供給される電力が,\(W=IV\;\si{(W)}\) であることを示せ.}
\vspace{20em}
\subsubsection{素子が電気抵抗 \(R\;\si{(\Omega)}\) のとき,\(W=RI^2\;\si{(W)}\) であることを示せ.(Jaule\('\)s law)}

\end{document}

