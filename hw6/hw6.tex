\documentclass[a4paper, 12pt]{bxjsarticle}
\usepackage{graphicx}
\usepackage{amsmath}
\usepackage{txfonts}
\usepackage{siunitx}
\usepackage{mathspec}
\setmainfont{IPAexMincho}
\setsansfont{IPAexGothic}
\XeTeXlinebreaklocale "ja"
\topmargin = 0mm
\pagestyle{empty}

\renewcommand{\thesubsection}{\arabic{subsection}.}
\renewcommand{\thesubsubsection}{(\alph{subsubsection})}

\begin {document}

\begin{center}
    \begin{huge}
        電磁気学 11/13 宿題
    \end{huge}
\end{center}

% p50, 例題1.7.3の類題
\subsection{p.51 図1.37のように,半径 \(a=0.3\si{[mm]}\),長さ \(l=1.0\si{[m]}\) の円柱状導体と,長さが同じで内半径 \(b=2.0\si{[mm]}\) の円筒状導体とを中心軸が一致するように配置して作られている円筒同軸ケーブルがある.\(V=5.0\si{[V]}\)の電圧が加わるとき,ケーブル内のエネルギーを求めよ.ただし導体間の間隔 \(b-a\) は導体の長さに比べて十分に小さく,電界は導体に挟まれた空間にのみ存在するとする.}


\newpage

% p61 1.13
\subsection{電気容量が\(C_1\)と\(C_2\)の2つのコンデンサーがある.初めに容量\(C_1\)のコンデンサーを電位差\(V\)で充電する.このとき容量\(C_1\)のコンデンサーに蓄えられる静電エネルギーを求めよ.次に,これを電源から切り放し,充電されていないもう一方のコンデンサーに並列につなぐと全体の静電エネルギーはどうなるか(p.61 図1.45(b) 参照).
また,\(C_1=0.01 \si{[\mu F]},\;C_2=0.25\si{[\mu F]}\),電圧\(V=50.0\si{[V]}\)の時の静電エネルギーを求めよ.}
\newpage

% テキストm22
\subsection{次式を証明せよ.ただし,\(f\)はスカラー,\(\boldsymbol{A}\)はベクトルとする.\[\nabla\cdot(f\boldsymbol{A})=f\nabla\cdot\boldsymbol{A}+\boldsymbol{A}\cdot\nabla f\]}
\end{document}

