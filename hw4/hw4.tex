\documentclass[a4j,12pt]{jsarticle}
\usepackage{amsmath}
\usepackage{txfonts}
\usepackage{siunitx}
\usepackage{bm}
\usepackage{upgreek}
\topmargin = 0mm
\pagestyle{empty}

\renewcommand{\thesubsection}{\arabic{subsection}.}
\renewcommand{\thesubsubsection}{(\alph{subsubsection})}

\begin {document}

\begin{center}
    \begin{LARGE}
        {\huge 電磁気学 10/30 宿題} 
    \end{LARGE}
\end{center}

% p38
\subsection{次の式を導け.}
\vspace*{-2em}
\begin{align}
    \bm{E} &= -\left(\frac{\partial \phi}{\partial r}\bm{e_r}+\frac{1}{r}\frac{\partial \phi}{\partial \theta}\bm{e_\theta}+\frac{1}{rsin\theta}\frac{\partial \phi}{\partial \varphi}\bm{e_\varphi}\right) \\
    \bm{E_r} &= -\frac{\partial \phi}{\partial r} = \frac{p\cos\theta}{2\pi\varepsilon_0r^3} \\
    \bm{E_\theta} &= -\frac{1}{r}\frac{\partial \phi}{\partial \theta} = \frac{p\sin\theta}{4\pi\varepsilon_0r^3} \\
    \bm{E_\varphi} &= -\frac{1}{r\sin\theta}\frac{\partial \phi}{\partial \varphi} = 0
\end{align}

\newpage

% p46, 問1.16
\subsection{半径\(a\)と\(b\)の2つの球状導体を十分に離して置き,この間を細い導線でつなぐ.これに総量\(Q\)の電荷を与えたときの両球の表面での電界を求めて,次の関係が成り立つことを示せ.}
\vspace*{-1em}
\begin{equation}
E=\frac{\sigma}{\varepsilon_0}
\end{equation}

\vspace{18em}

% p46, 問1.17
\subsection{半径\(a,\;b,\;c\)の3つの球状導体を十分に離して置き,これらの間を細い導線でつなぐ.これら全体に総量\(Q\)の電荷を与えたときに,個々の球の電位はどうなるか.また,個々の球の表面上の電荷密度はどうなるか.この時,\(b=10\si{[cm]},\;q=2.5\times10^{-10}\si{[C]},\;r=5\si{[cm]}\)での電位はどうなるか.}

\end{document}
