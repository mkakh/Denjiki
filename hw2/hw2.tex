\documentclass[a4j,12pt]{jsarticle}
\usepackage{amsmath}
\usepackage{txfonts}
\usepackage{siunitx}
\usepackage{bm}
\usepackage{upgreek}
\topmargin = 0mm
\pagestyle{empty}

\renewcommand{\thesubsection}{\arabic{subsection}.}
\renewcommand{\thesubsubsection}{(\alph{subsubsection})}

\begin {document}

\begin{center}
    \begin{LARGE}
        {\huge 電磁気学 10/16 宿題} 
    \end{LARGE}
\end{center}

% p14, 問1.5
\subsection{教科書p.13 図1.8(b)のように,\(x\)軸上に長さ\(2a\)にわたって電荷が一様に線密度\(\uplambda\)で分布している.\(x\)軸上で,原点\(\mathrm{O}\)から\(r\;(r>a)\)だけ離れた点\(\mathrm{P}\)での電界を求めよ.更に,\(\uplambda=1.5\times10^{-9}\si{[C/m]},\;r=10\si{[cm]}\)の時の\(\bm{E}\)を求めよ.}

\vspace{20em}

% p15, 問1.6
\subsection{無限に広い平面上に電荷が一様に面密度\(\upsigma\)で分布している.平面から距離\(z\)の点\(\mathrm{P}\)の電界を求めよ.更に,\(\upsigma=1.5\times10^{-10}\si{[C/m^2]}\)の時の\(\bm{E}\)の値を求めよ.}

\newpage

% p25, 問1.8
\subsection{半径\(a\)の無限に長い円筒上に電荷が一様に面密度\(\upsigma\)で分布している.円筒の中心軸から距離\(r\)だけ離れた点での電界を求めよ.}

\vspace{20em}

% p.25, 問1.9
\subsection{2枚の無限に広い平行な平面\(\mathrm{A}\)と\(\mathrm{B}\)がある.以下の場合についてまわりの電界を求めよ.}
\subsubsection{\(\mathrm{A}\)と\(\mathrm{B}\)上に電荷が一様に同じ面密度\(\upsigma\)で分布するとき.}
\vspace{12em}

\subsubsection{\(\mathrm{A}\)と\(\mathrm{B}\)上に電荷が一様にそれぞれ面密度\(\upsigma\)と\(-\upsigma\)で分布するとき.}
\newpage

% p.25, 問1.10
\subsection{半径\(a\)の球全体に総量\(Q\)の電荷が一様に分布している.球の中心\(\mathrm{O}\)から距離\(r\)の点\(\mathrm{P}\)での電界をガウスの法則を用いて求めよ.}
\end{document}
